
% Default to the notebook output style

    


% Inherit from the specified cell style.




    
\documentclass[11pt]{article}

    
    
    \usepackage[T1]{fontenc}
    % Nicer default font (+ math font) than Computer Modern for most use cases
    \usepackage{mathpazo}

    % Basic figure setup, for now with no caption control since it's done
    % automatically by Pandoc (which extracts ![](path) syntax from Markdown).
    \usepackage{graphicx}
    % We will generate all images so they have a width \maxwidth. This means
    % that they will get their normal width if they fit onto the page, but
    % are scaled down if they would overflow the margins.
    \makeatletter
    \def\maxwidth{\ifdim\Gin@nat@width>\linewidth\linewidth
    \else\Gin@nat@width\fi}
    \makeatother
    \let\Oldincludegraphics\includegraphics
    % Set max figure width to be 80% of text width, for now hardcoded.
    \renewcommand{\includegraphics}[1]{\Oldincludegraphics[width=.8\maxwidth]{#1}}
    % Ensure that by default, figures have no caption (until we provide a
    % proper Figure object with a Caption API and a way to capture that
    % in the conversion process - todo).
    \usepackage{caption}
    \DeclareCaptionLabelFormat{nolabel}{}
    \captionsetup{labelformat=nolabel}

    \usepackage{adjustbox} % Used to constrain images to a maximum size 
    \usepackage{xcolor} % Allow colors to be defined
    \usepackage{enumerate} % Needed for markdown enumerations to work
    \usepackage{geometry} % Used to adjust the document margins
    \usepackage{amsmath} % Equations
    \usepackage{amssymb} % Equations
    \usepackage{textcomp} % defines textquotesingle
    % Hack from http://tex.stackexchange.com/a/47451/13684:
    \AtBeginDocument{%
        \def\PYZsq{\textquotesingle}% Upright quotes in Pygmentized code
    }
    \usepackage{upquote} % Upright quotes for verbatim code
    \usepackage{eurosym} % defines \euro
    \usepackage[mathletters]{ucs} % Extended unicode (utf-8) support
    \usepackage[utf8x]{inputenc} % Allow utf-8 characters in the tex document
    \usepackage{fancyvrb} % verbatim replacement that allows latex
    \usepackage{grffile} % extends the file name processing of package graphics 
                         % to support a larger range 
    % The hyperref package gives us a pdf with properly built
    % internal navigation ('pdf bookmarks' for the table of contents,
    % internal cross-reference links, web links for URLs, etc.)
    \usepackage{hyperref}
    \usepackage{longtable} % longtable support required by pandoc >1.10
    \usepackage{booktabs}  % table support for pandoc > 1.12.2
    \usepackage[inline]{enumitem} % IRkernel/repr support (it uses the enumerate* environment)
    \usepackage[normalem]{ulem} % ulem is needed to support strikethroughs (\sout)
                                % normalem makes italics be italics, not underlines
    

    
    
    % Colors for the hyperref package
    \definecolor{urlcolor}{rgb}{0,.145,.698}
    \definecolor{linkcolor}{rgb}{.71,0.21,0.01}
    \definecolor{citecolor}{rgb}{.12,.54,.11}

    % ANSI colors
    \definecolor{ansi-black}{HTML}{3E424D}
    \definecolor{ansi-black-intense}{HTML}{282C36}
    \definecolor{ansi-red}{HTML}{E75C58}
    \definecolor{ansi-red-intense}{HTML}{B22B31}
    \definecolor{ansi-green}{HTML}{00A250}
    \definecolor{ansi-green-intense}{HTML}{007427}
    \definecolor{ansi-yellow}{HTML}{DDB62B}
    \definecolor{ansi-yellow-intense}{HTML}{B27D12}
    \definecolor{ansi-blue}{HTML}{208FFB}
    \definecolor{ansi-blue-intense}{HTML}{0065CA}
    \definecolor{ansi-magenta}{HTML}{D160C4}
    \definecolor{ansi-magenta-intense}{HTML}{A03196}
    \definecolor{ansi-cyan}{HTML}{60C6C8}
    \definecolor{ansi-cyan-intense}{HTML}{258F8F}
    \definecolor{ansi-white}{HTML}{C5C1B4}
    \definecolor{ansi-white-intense}{HTML}{A1A6B2}

    % commands and environments needed by pandoc snippets
    % extracted from the output of `pandoc -s`
    \providecommand{\tightlist}{%
      \setlength{\itemsep}{0pt}\setlength{\parskip}{0pt}}
    \DefineVerbatimEnvironment{Highlighting}{Verbatim}{commandchars=\\\{\}}
    % Add ',fontsize=\small' for more characters per line
    \newenvironment{Shaded}{}{}
    \newcommand{\KeywordTok}[1]{\textcolor[rgb]{0.00,0.44,0.13}{\textbf{{#1}}}}
    \newcommand{\DataTypeTok}[1]{\textcolor[rgb]{0.56,0.13,0.00}{{#1}}}
    \newcommand{\DecValTok}[1]{\textcolor[rgb]{0.25,0.63,0.44}{{#1}}}
    \newcommand{\BaseNTok}[1]{\textcolor[rgb]{0.25,0.63,0.44}{{#1}}}
    \newcommand{\FloatTok}[1]{\textcolor[rgb]{0.25,0.63,0.44}{{#1}}}
    \newcommand{\CharTok}[1]{\textcolor[rgb]{0.25,0.44,0.63}{{#1}}}
    \newcommand{\StringTok}[1]{\textcolor[rgb]{0.25,0.44,0.63}{{#1}}}
    \newcommand{\CommentTok}[1]{\textcolor[rgb]{0.38,0.63,0.69}{\textit{{#1}}}}
    \newcommand{\OtherTok}[1]{\textcolor[rgb]{0.00,0.44,0.13}{{#1}}}
    \newcommand{\AlertTok}[1]{\textcolor[rgb]{1.00,0.00,0.00}{\textbf{{#1}}}}
    \newcommand{\FunctionTok}[1]{\textcolor[rgb]{0.02,0.16,0.49}{{#1}}}
    \newcommand{\RegionMarkerTok}[1]{{#1}}
    \newcommand{\ErrorTok}[1]{\textcolor[rgb]{1.00,0.00,0.00}{\textbf{{#1}}}}
    \newcommand{\NormalTok}[1]{{#1}}
    
    % Additional commands for more recent versions of Pandoc
    \newcommand{\ConstantTok}[1]{\textcolor[rgb]{0.53,0.00,0.00}{{#1}}}
    \newcommand{\SpecialCharTok}[1]{\textcolor[rgb]{0.25,0.44,0.63}{{#1}}}
    \newcommand{\VerbatimStringTok}[1]{\textcolor[rgb]{0.25,0.44,0.63}{{#1}}}
    \newcommand{\SpecialStringTok}[1]{\textcolor[rgb]{0.73,0.40,0.53}{{#1}}}
    \newcommand{\ImportTok}[1]{{#1}}
    \newcommand{\DocumentationTok}[1]{\textcolor[rgb]{0.73,0.13,0.13}{\textit{{#1}}}}
    \newcommand{\AnnotationTok}[1]{\textcolor[rgb]{0.38,0.63,0.69}{\textbf{\textit{{#1}}}}}
    \newcommand{\CommentVarTok}[1]{\textcolor[rgb]{0.38,0.63,0.69}{\textbf{\textit{{#1}}}}}
    \newcommand{\VariableTok}[1]{\textcolor[rgb]{0.10,0.09,0.49}{{#1}}}
    \newcommand{\ControlFlowTok}[1]{\textcolor[rgb]{0.00,0.44,0.13}{\textbf{{#1}}}}
    \newcommand{\OperatorTok}[1]{\textcolor[rgb]{0.40,0.40,0.40}{{#1}}}
    \newcommand{\BuiltInTok}[1]{{#1}}
    \newcommand{\ExtensionTok}[1]{{#1}}
    \newcommand{\PreprocessorTok}[1]{\textcolor[rgb]{0.74,0.48,0.00}{{#1}}}
    \newcommand{\AttributeTok}[1]{\textcolor[rgb]{0.49,0.56,0.16}{{#1}}}
    \newcommand{\InformationTok}[1]{\textcolor[rgb]{0.38,0.63,0.69}{\textbf{\textit{{#1}}}}}
    \newcommand{\WarningTok}[1]{\textcolor[rgb]{0.38,0.63,0.69}{\textbf{\textit{{#1}}}}}
    
    
    % Define a nice break command that doesn't care if a line doesn't already
    % exist.
    \def\br{\hspace*{\fill} \\* }
    % Math Jax compatability definitions
    \def\gt{>}
    \def\lt{<}
    % Document parameters
    \title{HW3}
    
    
    

    % Pygments definitions
    
\makeatletter
\def\PY@reset{\let\PY@it=\relax \let\PY@bf=\relax%
    \let\PY@ul=\relax \let\PY@tc=\relax%
    \let\PY@bc=\relax \let\PY@ff=\relax}
\def\PY@tok#1{\csname PY@tok@#1\endcsname}
\def\PY@toks#1+{\ifx\relax#1\empty\else%
    \PY@tok{#1}\expandafter\PY@toks\fi}
\def\PY@do#1{\PY@bc{\PY@tc{\PY@ul{%
    \PY@it{\PY@bf{\PY@ff{#1}}}}}}}
\def\PY#1#2{\PY@reset\PY@toks#1+\relax+\PY@do{#2}}

\expandafter\def\csname PY@tok@w\endcsname{\def\PY@tc##1{\textcolor[rgb]{0.73,0.73,0.73}{##1}}}
\expandafter\def\csname PY@tok@c\endcsname{\let\PY@it=\textit\def\PY@tc##1{\textcolor[rgb]{0.25,0.50,0.50}{##1}}}
\expandafter\def\csname PY@tok@cp\endcsname{\def\PY@tc##1{\textcolor[rgb]{0.74,0.48,0.00}{##1}}}
\expandafter\def\csname PY@tok@k\endcsname{\let\PY@bf=\textbf\def\PY@tc##1{\textcolor[rgb]{0.00,0.50,0.00}{##1}}}
\expandafter\def\csname PY@tok@kp\endcsname{\def\PY@tc##1{\textcolor[rgb]{0.00,0.50,0.00}{##1}}}
\expandafter\def\csname PY@tok@kt\endcsname{\def\PY@tc##1{\textcolor[rgb]{0.69,0.00,0.25}{##1}}}
\expandafter\def\csname PY@tok@o\endcsname{\def\PY@tc##1{\textcolor[rgb]{0.40,0.40,0.40}{##1}}}
\expandafter\def\csname PY@tok@ow\endcsname{\let\PY@bf=\textbf\def\PY@tc##1{\textcolor[rgb]{0.67,0.13,1.00}{##1}}}
\expandafter\def\csname PY@tok@nb\endcsname{\def\PY@tc##1{\textcolor[rgb]{0.00,0.50,0.00}{##1}}}
\expandafter\def\csname PY@tok@nf\endcsname{\def\PY@tc##1{\textcolor[rgb]{0.00,0.00,1.00}{##1}}}
\expandafter\def\csname PY@tok@nc\endcsname{\let\PY@bf=\textbf\def\PY@tc##1{\textcolor[rgb]{0.00,0.00,1.00}{##1}}}
\expandafter\def\csname PY@tok@nn\endcsname{\let\PY@bf=\textbf\def\PY@tc##1{\textcolor[rgb]{0.00,0.00,1.00}{##1}}}
\expandafter\def\csname PY@tok@ne\endcsname{\let\PY@bf=\textbf\def\PY@tc##1{\textcolor[rgb]{0.82,0.25,0.23}{##1}}}
\expandafter\def\csname PY@tok@nv\endcsname{\def\PY@tc##1{\textcolor[rgb]{0.10,0.09,0.49}{##1}}}
\expandafter\def\csname PY@tok@no\endcsname{\def\PY@tc##1{\textcolor[rgb]{0.53,0.00,0.00}{##1}}}
\expandafter\def\csname PY@tok@nl\endcsname{\def\PY@tc##1{\textcolor[rgb]{0.63,0.63,0.00}{##1}}}
\expandafter\def\csname PY@tok@ni\endcsname{\let\PY@bf=\textbf\def\PY@tc##1{\textcolor[rgb]{0.60,0.60,0.60}{##1}}}
\expandafter\def\csname PY@tok@na\endcsname{\def\PY@tc##1{\textcolor[rgb]{0.49,0.56,0.16}{##1}}}
\expandafter\def\csname PY@tok@nt\endcsname{\let\PY@bf=\textbf\def\PY@tc##1{\textcolor[rgb]{0.00,0.50,0.00}{##1}}}
\expandafter\def\csname PY@tok@nd\endcsname{\def\PY@tc##1{\textcolor[rgb]{0.67,0.13,1.00}{##1}}}
\expandafter\def\csname PY@tok@s\endcsname{\def\PY@tc##1{\textcolor[rgb]{0.73,0.13,0.13}{##1}}}
\expandafter\def\csname PY@tok@sd\endcsname{\let\PY@it=\textit\def\PY@tc##1{\textcolor[rgb]{0.73,0.13,0.13}{##1}}}
\expandafter\def\csname PY@tok@si\endcsname{\let\PY@bf=\textbf\def\PY@tc##1{\textcolor[rgb]{0.73,0.40,0.53}{##1}}}
\expandafter\def\csname PY@tok@se\endcsname{\let\PY@bf=\textbf\def\PY@tc##1{\textcolor[rgb]{0.73,0.40,0.13}{##1}}}
\expandafter\def\csname PY@tok@sr\endcsname{\def\PY@tc##1{\textcolor[rgb]{0.73,0.40,0.53}{##1}}}
\expandafter\def\csname PY@tok@ss\endcsname{\def\PY@tc##1{\textcolor[rgb]{0.10,0.09,0.49}{##1}}}
\expandafter\def\csname PY@tok@sx\endcsname{\def\PY@tc##1{\textcolor[rgb]{0.00,0.50,0.00}{##1}}}
\expandafter\def\csname PY@tok@m\endcsname{\def\PY@tc##1{\textcolor[rgb]{0.40,0.40,0.40}{##1}}}
\expandafter\def\csname PY@tok@gh\endcsname{\let\PY@bf=\textbf\def\PY@tc##1{\textcolor[rgb]{0.00,0.00,0.50}{##1}}}
\expandafter\def\csname PY@tok@gu\endcsname{\let\PY@bf=\textbf\def\PY@tc##1{\textcolor[rgb]{0.50,0.00,0.50}{##1}}}
\expandafter\def\csname PY@tok@gd\endcsname{\def\PY@tc##1{\textcolor[rgb]{0.63,0.00,0.00}{##1}}}
\expandafter\def\csname PY@tok@gi\endcsname{\def\PY@tc##1{\textcolor[rgb]{0.00,0.63,0.00}{##1}}}
\expandafter\def\csname PY@tok@gr\endcsname{\def\PY@tc##1{\textcolor[rgb]{1.00,0.00,0.00}{##1}}}
\expandafter\def\csname PY@tok@ge\endcsname{\let\PY@it=\textit}
\expandafter\def\csname PY@tok@gs\endcsname{\let\PY@bf=\textbf}
\expandafter\def\csname PY@tok@gp\endcsname{\let\PY@bf=\textbf\def\PY@tc##1{\textcolor[rgb]{0.00,0.00,0.50}{##1}}}
\expandafter\def\csname PY@tok@go\endcsname{\def\PY@tc##1{\textcolor[rgb]{0.53,0.53,0.53}{##1}}}
\expandafter\def\csname PY@tok@gt\endcsname{\def\PY@tc##1{\textcolor[rgb]{0.00,0.27,0.87}{##1}}}
\expandafter\def\csname PY@tok@err\endcsname{\def\PY@bc##1{\setlength{\fboxsep}{0pt}\fcolorbox[rgb]{1.00,0.00,0.00}{1,1,1}{\strut ##1}}}
\expandafter\def\csname PY@tok@kc\endcsname{\let\PY@bf=\textbf\def\PY@tc##1{\textcolor[rgb]{0.00,0.50,0.00}{##1}}}
\expandafter\def\csname PY@tok@kd\endcsname{\let\PY@bf=\textbf\def\PY@tc##1{\textcolor[rgb]{0.00,0.50,0.00}{##1}}}
\expandafter\def\csname PY@tok@kn\endcsname{\let\PY@bf=\textbf\def\PY@tc##1{\textcolor[rgb]{0.00,0.50,0.00}{##1}}}
\expandafter\def\csname PY@tok@kr\endcsname{\let\PY@bf=\textbf\def\PY@tc##1{\textcolor[rgb]{0.00,0.50,0.00}{##1}}}
\expandafter\def\csname PY@tok@bp\endcsname{\def\PY@tc##1{\textcolor[rgb]{0.00,0.50,0.00}{##1}}}
\expandafter\def\csname PY@tok@fm\endcsname{\def\PY@tc##1{\textcolor[rgb]{0.00,0.00,1.00}{##1}}}
\expandafter\def\csname PY@tok@vc\endcsname{\def\PY@tc##1{\textcolor[rgb]{0.10,0.09,0.49}{##1}}}
\expandafter\def\csname PY@tok@vg\endcsname{\def\PY@tc##1{\textcolor[rgb]{0.10,0.09,0.49}{##1}}}
\expandafter\def\csname PY@tok@vi\endcsname{\def\PY@tc##1{\textcolor[rgb]{0.10,0.09,0.49}{##1}}}
\expandafter\def\csname PY@tok@vm\endcsname{\def\PY@tc##1{\textcolor[rgb]{0.10,0.09,0.49}{##1}}}
\expandafter\def\csname PY@tok@sa\endcsname{\def\PY@tc##1{\textcolor[rgb]{0.73,0.13,0.13}{##1}}}
\expandafter\def\csname PY@tok@sb\endcsname{\def\PY@tc##1{\textcolor[rgb]{0.73,0.13,0.13}{##1}}}
\expandafter\def\csname PY@tok@sc\endcsname{\def\PY@tc##1{\textcolor[rgb]{0.73,0.13,0.13}{##1}}}
\expandafter\def\csname PY@tok@dl\endcsname{\def\PY@tc##1{\textcolor[rgb]{0.73,0.13,0.13}{##1}}}
\expandafter\def\csname PY@tok@s2\endcsname{\def\PY@tc##1{\textcolor[rgb]{0.73,0.13,0.13}{##1}}}
\expandafter\def\csname PY@tok@sh\endcsname{\def\PY@tc##1{\textcolor[rgb]{0.73,0.13,0.13}{##1}}}
\expandafter\def\csname PY@tok@s1\endcsname{\def\PY@tc##1{\textcolor[rgb]{0.73,0.13,0.13}{##1}}}
\expandafter\def\csname PY@tok@mb\endcsname{\def\PY@tc##1{\textcolor[rgb]{0.40,0.40,0.40}{##1}}}
\expandafter\def\csname PY@tok@mf\endcsname{\def\PY@tc##1{\textcolor[rgb]{0.40,0.40,0.40}{##1}}}
\expandafter\def\csname PY@tok@mh\endcsname{\def\PY@tc##1{\textcolor[rgb]{0.40,0.40,0.40}{##1}}}
\expandafter\def\csname PY@tok@mi\endcsname{\def\PY@tc##1{\textcolor[rgb]{0.40,0.40,0.40}{##1}}}
\expandafter\def\csname PY@tok@il\endcsname{\def\PY@tc##1{\textcolor[rgb]{0.40,0.40,0.40}{##1}}}
\expandafter\def\csname PY@tok@mo\endcsname{\def\PY@tc##1{\textcolor[rgb]{0.40,0.40,0.40}{##1}}}
\expandafter\def\csname PY@tok@ch\endcsname{\let\PY@it=\textit\def\PY@tc##1{\textcolor[rgb]{0.25,0.50,0.50}{##1}}}
\expandafter\def\csname PY@tok@cm\endcsname{\let\PY@it=\textit\def\PY@tc##1{\textcolor[rgb]{0.25,0.50,0.50}{##1}}}
\expandafter\def\csname PY@tok@cpf\endcsname{\let\PY@it=\textit\def\PY@tc##1{\textcolor[rgb]{0.25,0.50,0.50}{##1}}}
\expandafter\def\csname PY@tok@c1\endcsname{\let\PY@it=\textit\def\PY@tc##1{\textcolor[rgb]{0.25,0.50,0.50}{##1}}}
\expandafter\def\csname PY@tok@cs\endcsname{\let\PY@it=\textit\def\PY@tc##1{\textcolor[rgb]{0.25,0.50,0.50}{##1}}}

\def\PYZbs{\char`\\}
\def\PYZus{\char`\_}
\def\PYZob{\char`\{}
\def\PYZcb{\char`\}}
\def\PYZca{\char`\^}
\def\PYZam{\char`\&}
\def\PYZlt{\char`\<}
\def\PYZgt{\char`\>}
\def\PYZsh{\char`\#}
\def\PYZpc{\char`\%}
\def\PYZdl{\char`\$}
\def\PYZhy{\char`\-}
\def\PYZsq{\char`\'}
\def\PYZdq{\char`\"}
\def\PYZti{\char`\~}
% for compatibility with earlier versions
\def\PYZat{@}
\def\PYZlb{[}
\def\PYZrb{]}
\makeatother


    % Exact colors from NB
    \definecolor{incolor}{rgb}{0.0, 0.0, 0.5}
    \definecolor{outcolor}{rgb}{0.545, 0.0, 0.0}



    
    % Prevent overflowing lines due to hard-to-break entities
    \sloppy 
    % Setup hyperref package
    \hypersetup{
      breaklinks=true,  % so long urls are correctly broken across lines
      colorlinks=true,
      urlcolor=urlcolor,
      linkcolor=linkcolor,
      citecolor=citecolor,
      }
    % Slightly bigger margins than the latex defaults
    
    \geometry{verbose,tmargin=1in,bmargin=1in,lmargin=1in,rmargin=1in}
    
    

    \begin{document}
    
    
    \maketitle
    
    

    
    \begin{Verbatim}[commandchars=\\\{\}]
{\color{incolor}In [{\color{incolor}210}]:} \PY{k+kn}{library}\PY{p}{(}ggplot2\PY{p}{)}
          \PY{k+kn}{library}\PY{p}{(}dplyr\PY{p}{)}
          \PY{k+kn}{library}\PY{p}{(}magrittr\PY{p}{)}
          \PY{k+kn}{library}\PY{p}{(}tidyr\PY{p}{)}
          \PY{k+kn}{library}\PY{p}{(}stringr\PY{p}{)}
          \PY{k+kn}{library}\PY{p}{(}repr\PY{p}{)}
          survey \PY{o}{\PYZlt{}\PYZhy{}} read.csv\PY{p}{(}\PY{l+s}{\PYZsq{}}\PY{l+s}{./2010 Survey \PYZhy{} Student Data \PYZhy{} 437 responses.csv\PYZsq{}}\PY{p}{)}
          \PY{k+kp}{names}\PY{p}{(}survey\PY{p}{)}\PY{p}{[}\PY{l+m}{63}\PY{p}{]} \PY{o}{\PYZlt{}\PYZhy{}} \PY{l+s}{\PYZsq{}}\PY{l+s}{totfriends\PYZsq{}}
\end{Verbatim}


    Robin Dunbar a famous anthropologist argued that humans only have the
capacity to keep track of so many people at a time. He argued that the
maximum number of people humans could keep track of is 150. This number
is called Dunbar's number (Links to an external site.). Some have
interpreted this to mean that any individual person has on average about
150 friends.

Some technologists have argued that social media and modern information
technology allow individuals to keep track of more people. For example,
Facebook algorithmically organizes and presents information about our
Facebook friends in a centralized feed lowering the costs associated
with keeping in contact with all these people.

Fortunately we can evaluate this claim because our colleague Prof. Vitak
has collected data on Facebook usage among college students at a large
university. Using this sample dataPreview the documentView in a new
window and the codebookPreview the documentView in a new window which
explains the variables evaluate the technologists's claim.

Question 1: Using the theoretical population mean and the sample
provided conduct a one-sample t-test (with significance level alpha =
0.05) to determine whether individuals have more friends than would be
expected by Dunbar's number. (R Hints: 1) convert .xls to .csv by saving
as .csv in Excel this will make it easier to read in to R; 2) relabel
the columns that you are going to use to one word labels with no spaces
in Excel, this will make it easier to refer to them in R e.g. change
"How old are you?" to "age" you don't need to do this with all columns,
just the ones you plan to use.)

\begin{enumerate}
\def\labelenumi{\alph{enumi}.}
\tightlist
\item
  Graph the total Facebook friend values for the sample. What is the
  shape?
\end{enumerate}

    \begin{Verbatim}[commandchars=\\\{\}]
{\color{incolor}In [{\color{incolor}244}]:} \PY{k+kp}{options}\PY{p}{(}repr.plot.width\PY{o}{=}\PY{l+m}{4}\PY{p}{,} repr.plot.height\PY{o}{=}\PY{l+m}{4}\PY{p}{)}
          hist\PY{p}{(}survey\PY{o}{\PYZdl{}}totfriends\PY{p}{,} xlab\PY{o}{=}\PY{l+s}{\PYZdq{}}\PY{l+s}{Total Friends\PYZdq{}}\PY{p}{,} main\PY{o}{=}\PY{l+s}{\PYZdq{}}\PY{l+s}{\PYZdq{}}\PY{p}{)}
          qqnorm\PY{p}{(}survey\PY{o}{\PYZdl{}}totfriends\PY{p}{,} main\PY{o}{=}\PY{l+s}{\PYZdq{}}\PY{l+s}{\PYZdq{}}\PY{p}{)}\PY{p}{;} qqline\PY{p}{(}survey\PY{o}{\PYZdl{}}totfriends\PY{p}{,} col \PY{o}{=} \PY{l+m}{2}\PY{p}{)}
\end{Verbatim}


    \begin{center}
    \adjustimage{max size={0.9\linewidth}{0.9\paperheight}}{output_2_0.png}
    \end{center}
    { \hspace*{\fill} \\}
    
    \begin{center}
    \adjustimage{max size={0.9\linewidth}{0.9\paperheight}}{output_2_1.png}
    \end{center}
    { \hspace*{\fill} \\}
    
    \begin{enumerate}
\def\labelenumi{\alph{enumi}.}
\setcounter{enumi}{1}
\tightlist
\item
  State your null and alternative hypotheses.
\end{enumerate}

    \begin{Verbatim}[commandchars=\\\{\}]
{\color{incolor}In [{\color{incolor}149}]:} N \PY{o}{\PYZlt{}\PYZhy{}} \PY{k+kp}{nrow}\PY{p}{(}survey\PY{p}{)}
          xbar \PY{o}{\PYZlt{}\PYZhy{}} \PY{k+kp}{mean}\PY{p}{(}survey\PY{o}{\PYZdl{}}totfriends\PY{p}{,} na.rm\PY{o}{=}\PY{n+nb+bp}{T}\PY{p}{)}
          
          mu \PY{o}{\PYZlt{}\PYZhy{}} 𝝁 \PY{o}{\PYZlt{}\PYZhy{}} dunbar \PY{o}{\PYZlt{}\PYZhy{}} \PY{l+m}{150}
          IRdisplay\PY{o}{::}display\PYZus{}markdown\PY{p}{(}glue\PY{o}{::}glue\PY{p}{(}\PY{l+s}{\PYZsq{}}
          \PY{l+s}{Hypothesis: Students have more Facebook friends than predicted by Dunbar\PYZbs{}\PYZsq{}s number}
          
          \PY{l+s}{\PYZdl{}\PYZdl{}H\PYZus{}0 : \PYZbs{}\PYZbs{}bar\PYZob{}\PYZob{}x\PYZcb{}\PYZcb{} = \PYZbs{}\PYZbs{}mu\PYZdl{}\PYZdl{}}
          \PY{l+s}{\PYZdl{}H\PYZus{}A : \PYZbs{}\PYZbs{}bar\PYZob{}\PYZob{}x\PYZcb{}\PYZcb{} \PYZbs{}\PYZbs{}ne \PYZbs{}\PYZbs{}mu \PYZbs{}\PYZbs{}quad\PYZdl{} two\PYZhy{}tailed for more conservative evaluation}
          
          \PY{l+s}{\PYZdl{}\PYZbs{}\PYZbs{}mu: \PYZob{}fmt(𝝁)\PYZcb{} \PYZbs{}\PYZbs{}quad\PYZdl{} Dunbar\PYZbs{}\PYZsq{}s number}
          \PY{l+s}{\PYZdl{}\PYZdl{}\PYZbs{}\PYZbs{}bar\PYZob{}\PYZob{}x\PYZcb{}\PYZcb{}: \PYZob{}fmt(xbar)\PYZcb{}\PYZdl{}\PYZdl{}}
          \PY{l+s}{\PYZsq{}}\PY{p}{)}\PY{p}{)}
\end{Verbatim}


    Hypothesis: Students have more Facebook friends than predicted by
Dunbar's number

\[H_0 : \bar{x} = \mu\] \(H_A : \bar{x} \ne \mu \quad\) two-tailed for
more conservative evaluation

\(\mu: 150 \quad\) Dunbar's number \[\bar{x}: 433\]

    
    \begin{Verbatim}[commandchars=\\\{\}]
{\color{incolor}In [{\color{incolor}152}]:} IRdisplay\PY{o}{::}display\PYZus{}markdown\PY{p}{(}glue\PY{o}{::}glue\PY{p}{(}\PY{l+s}{\PYZsq{}}
          \PY{l+s}{df: \PYZob{}fmt(df)\PYZcb{}\PYZbs{}n}
          \PY{l+s}{t\PYZhy{}score: \PYZdl{}\PYZbs{}\PYZbs{}dfrac\PYZob{}\PYZob{}\PYZbs{}\PYZbs{}bar\PYZob{}\PYZob{}x\PYZcb{}\PYZcb{} \PYZhy{} \PYZbs{}\PYZbs{}mu\PYZcb{}\PYZcb{}\PYZob{}\PYZob{}s\PYZbs{}\PYZbs{}over\PYZob{}\PYZob{}\PYZbs{}\PYZbs{}sqrt\PYZob{}\PYZob{}n\PYZcb{}\PYZcb{}\PYZcb{}\PYZcb{}\PYZcb{}\PYZcb{}\PYZdl{} =}
          \PY{l+s}{\PYZdl{}\PYZbs{}\PYZbs{}dfrac\PYZob{}\PYZob{} \PYZob{}fmt(xbar)\PYZcb{} \PYZhy{} \PYZob{}mu\PYZcb{}\PYZcb{}\PYZob{}\PYZob{}\PYZob{}fmt(s,2)\PYZcb{}\PYZbs{}\PYZbs{}over\PYZob{}\PYZob{}\PYZbs{}\PYZbs{}sqrt\PYZob{}\PYZob{}\PYZob{}n\PYZcb{}\PYZcb{}\PYZcb{}\PYZcb{}\PYZcb{}\PYZcb{}\PYZcb{}\PYZdl{} = \PYZob{}fmt(tscore,2)\PYZcb{}}
          \PY{l+s}{\PYZsq{}}\PY{p}{)}\PY{p}{)}
\end{Verbatim}


    df: 436

t-score: \(\dfrac{\bar{x} - \mu}{s\over{\sqrt{n}}}\) =
\(\dfrac{ 433 - 150}{275.34\over{\sqrt{474}}}\) = 21.49

    
    \begin{enumerate}
\def\labelenumi{\alph{enumi}.}
\setcounter{enumi}{2}
\tightlist
\item
  Calculate your t-statistic
\end{enumerate}

    \begin{Verbatim}[commandchars=\\\{\}]
{\color{incolor}In [{\color{incolor}201}]:} s \PY{o}{\PYZlt{}\PYZhy{}} sd\PY{p}{(}survey\PY{o}{\PYZdl{}}totfriends\PY{p}{,} na.rm\PY{o}{=}\PY{n+nb+bp}{T}\PY{p}{)}
          se \PY{o}{\PYZlt{}\PYZhy{}} s\PY{o}{/}\PY{k+kp}{sqrt}\PY{p}{(}N\PY{p}{)}
          
          tscore \PY{o}{\PYZlt{}\PYZhy{}} \PY{p}{(}xbar \PY{o}{\PYZhy{}} mu\PY{p}{)} \PY{o}{/} \PY{p}{(}s\PY{o}{/}\PY{k+kp}{sqrt}\PY{p}{(}N\PY{p}{)}\PY{p}{)} \PY{c+c1}{\PYZsh{}  \PYZsh{} week 6 slides, p. 23}
          
          IRdisplay\PY{o}{::}display\PYZus{}markdown\PY{p}{(}glue\PY{o}{::}glue\PY{p}{(}\PY{l+s}{\PYZsq{}}
          \PY{l+s}{t\PYZhy{}score: \PYZdl{}T\PYZus{}\PYZob{}\PYZob{}df\PYZcb{}\PYZcb{} = \PYZbs{}\PYZbs{}dfrac\PYZob{}\PYZob{}point\PYZbs{}\PYZbs{};estimate \PYZhy{} null\PYZbs{}\PYZbs{};value\PYZcb{}\PYZcb{}\PYZob{}\PYZob{}SE\PYZcb{}\PYZcb{} \PYZbs{}\PYZbs{}quad\PYZdl{} week 6 slides, pp. 23, 51}
          
          \PY{l+s}{SE: \PYZdl{}\PYZbs{}\PYZbs{}dfrac\PYZob{}\PYZob{}s\PYZcb{}\PYZcb{}\PYZob{}\PYZob{}\PYZbs{}\PYZbs{}sqrt\PYZob{}\PYZob{}n\PYZcb{}\PYZcb{}\PYZcb{}\PYZdl{} = \PYZdl{}\PYZbs{}\PYZbs{}dfrac\PYZob{}\PYZob{}\PYZob{}fmt(s)\PYZcb{}\PYZcb{}\PYZob{}\PYZob{}\PYZbs{}\PYZbs{}sqrt\PYZob{}\PYZob{}\PYZob{}fmt(n,2)\PYZcb{}\PYZcb{}\PYZcb{}\PYZcb{}\PYZcb{}\PYZdl{} = \PYZob{}fmt(se,2)\PYZcb{}}
          
          \PY{l+s}{\PYZdl{}T = \PYZbs{}\PYZbs{}dfrac\PYZob{}\PYZob{}\PYZbs{}\PYZbs{}bar\PYZob{}\PYZob{}x\PYZcb{}\PYZcb{} \PYZhy{} \PYZbs{}\PYZbs{}mu\PYZcb{}\PYZcb{}\PYZob{}\PYZob{}\PYZob{}\PYZob{}SE\PYZcb{}\PYZcb{}\PYZcb{}\PYZdl{} }
          \PY{l+s}{\PYZdl{}\PYZbs{}\PYZbs{}dfrac\PYZob{}\PYZob{} \PYZob{}fmt(xbar)\PYZcb{} \PYZhy{} \PYZob{}mu\PYZcb{}\PYZcb{}\PYZcb{}\PYZob{}\PYZob{}\PYZob{}fmt(se,2)\PYZcb{}\PYZcb{}\PYZcb{} = \PYZob{}fmt(tscore,2)\PYZcb{}\PYZdl{}}
          
          \PY{l+s}{\PYZsq{}}\PY{p}{)}\PY{p}{)}
\end{Verbatim}


    t-score: \(T_{df} = \dfrac{point\;estimate - null\;value}{SE} \quad\)
week 6 slides, pp.~23, 51

SE: \(\dfrac{s}{\sqrt{n}}\) = \(\dfrac{275}{\sqrt{474}}\) = 13.17

\(T = \dfrac{\bar{x} - \mu}{{SE}}\)
\(\dfrac{ 433 - 150}{13.17} = 21.49\)

    
    \begin{enumerate}
\def\labelenumi{\alph{enumi}.}
\setcounter{enumi}{3}
\tightlist
\item
  Look up the probability for the t-statistic
\end{enumerate}

    \begin{Verbatim}[commandchars=\\\{\}]
{\color{incolor}In [{\color{incolor}162}]:} df \PY{o}{\PYZlt{}\PYZhy{}} N \PY{o}{\PYZhy{}} \PY{l+m}{1}
          p\PYZus{}tscore \PY{o}{\PYZlt{}\PYZhy{}} pt\PY{p}{(}tscore\PY{p}{,} df\PY{p}{,} lower.tail \PY{o}{=} \PY{n+nb+bp}{F}\PY{p}{)}
          
          IRdisplay\PY{o}{::}display\PYZus{}markdown\PY{p}{(}glue\PY{o}{::}glue\PY{p}{(}\PY{l+s}{\PYZsq{}}
          \PY{l+s}{probability of t\PYZhy{}statistic: *pt(tscore, df, lower.tail = F)* = \PYZob{}p\PYZus{}tscore\PYZcb{}}
          \PY{l+s}{\PYZsq{}}\PY{p}{)}\PY{p}{)}
\end{Verbatim}


    probability of t-statistic: \emph{pt(tscore, df, lower.tail = F)} =
1.13892675278444e-70

    
    \begin{Verbatim}[commandchars=\\\{\}]
{\color{incolor}In [{\color{incolor}209}]:} t.test\PY{p}{(}survey\PY{o}{\PYZdl{}}totfriends\PY{p}{,} mu\PY{o}{=}mu\PY{p}{)}
\end{Verbatim}


    
    \begin{verbatim}

	One Sample t-test

data:  survey$totfriends
t = 20.271, df = 388, p-value < 2.2e-16
alternative hypothesis: true mean is not equal to 150
95 percent confidence interval:
 405.5476 460.4421
sample estimates:
mean of x 
 432.9949 

    \end{verbatim}

    
    \begin{enumerate}
\def\labelenumi{\alph{enumi}.}
\setcounter{enumi}{4}
\tightlist
\item
  What is the effect size?
\end{enumerate}

    \begin{Verbatim}[commandchars=\\\{\}]
{\color{incolor}In [{\color{incolor}208}]:} effect\PYZus{}size \PY{o}{\PYZlt{}\PYZhy{}} cohend \PY{o}{\PYZlt{}\PYZhy{}} \PY{p}{(}xbar \PY{o}{\PYZhy{}} mu\PY{p}{)} \PY{o}{/} s
          critical\PYZus{}t \PY{o}{\PYZlt{}\PYZhy{}} tstar \PY{o}{\PYZlt{}\PYZhy{}} qt\PY{p}{(}\PY{l+m}{1} \PY{o}{\PYZhy{}} alpha\PY{o}{/}\PY{l+m}{2}\PY{p}{,} df\PY{p}{)}
          IRdisplay\PY{o}{::}display\PYZus{}markdown\PY{p}{(}glue\PY{o}{::}glue\PY{p}{(}\PY{l+s}{\PYZsq{}}
          \PY{l+s}{Cohen\PYZbs{}\PYZsq{}s d = \PYZdl{}\PYZbs{}\PYZbs{}dfrac\PYZob{}\PYZob{}point\PYZbs{}\PYZbs{};estimate \PYZhy{} null\PYZbs{}\PYZbs{};value\PYZcb{}\PYZcb{}\PYZob{}\PYZob{}s\PYZcb{}\PYZcb{} = \PYZdl{}}
          \PY{l+s}{\PYZdl{}\PYZbs{}\PYZbs{}dfrac\PYZob{}\PYZob{} \PYZob{}fmt(xbar)\PYZcb{} \PYZhy{} \PYZob{}mu\PYZcb{}\PYZcb{}\PYZcb{}\PYZob{}\PYZob{}\PYZob{}fmt(s,2)\PYZcb{}\PYZcb{}\PYZcb{} = \PYZob{}fmt(cohend,2)\PYZcb{}\PYZdl{} }
          \PY{l+s}{\PYZsq{}}\PY{p}{)}\PY{p}{)}
\end{Verbatim}


    Cohen's d = \$\dfrac{point\;estimate - null\;value}{s} = \$
\(\dfrac{ 433 - 150}{275.34} = 1.03\)

    
    \begin{enumerate}
\def\labelenumi{\alph{enumi}.}
\setcounter{enumi}{5}
\tightlist
\item
  What would you conclude?
\end{enumerate}

\textbf{There's a very high probability (p \textless{} 0.001) that the
mean number of Facebook friends counted by students in a large
university is greater than Dunbar's number, with an effect size of about
1 standard deviation (Cohen's d = 1.03), which is large.}

\begin{enumerate}
\def\labelenumi{\alph{enumi}.}
\setcounter{enumi}{6}
\tightlist
\item
  What are the limitations, if any?
\end{enumerate}

**The sample is students at a large university in the U.S. Results may
not generalize to other populations.

    Question 2: Are individuals who have 150 or more friends on Facebook
using Facebook in a fundamentally different way from those that have
fewer than 150 friends? Use two sample t-tests with a significance level
of 0.05 to compare Facebook users who have 150 Facebook friends or more
to those with fewer than 150 Facebook friends for the 6 items that make
up question 28 "How likely are you to use Facebook to do the following
things, either now or in the future?". Feel free to do the calculations
using t.test in R. Treat these Likert items as interval variables.

Hints: 1) Use the ifelse function in R to create a new variable binary
variable to categorize users with 150 or more Facebook friends from
those with fewer than 150 Facebook friends 2) You will need to run 6 t
tests 3) for this problem you don't need to write out the null and
alternative hypotheses (although you should know what these are!), 4)
for this problem you don't need to calculate effect sizes (although if
you have time and want the practice feel free to do so for an extra
challenge).

\begin{enumerate}
\def\labelenumi{\alph{enumi})}
\tightlist
\item
  Calculate 6 independent t-tests and organize results into a table with
  independent variable, dependent variable, and results of the
  statistical tests (i.e. t-statistic, degrees of freedom, and p-value)
\end{enumerate}

    \begin{Verbatim}[commandchars=\\\{\}]
{\color{incolor}In [{\color{incolor}226}]:} survey\PY{o}{\PYZdl{}}manyfriends \PY{o}{\PYZlt{}\PYZhy{}} survey\PY{o}{\PYZdl{}}totfriends \PY{o}{\PYZgt{}=} \PY{l+m}{150}
          
          \PY{c+c1}{\PYZsh{}https://stackoverflow.com/a/15223238/1368860}
          
          tests \PY{o}{\PYZlt{}\PYZhy{}} \PY{k+kp}{lapply}\PY{p}{(}\PY{k+kp}{names}\PY{p}{(}survey\PY{p}{)}\PY{p}{[}\PY{l+m}{87}\PY{o}{:}\PY{l+m}{92}\PY{p}{]}\PY{p}{,}\PY{k+kr}{function}\PY{p}{(}x\PY{p}{)} t.test\PY{p}{(}as.formula\PY{p}{(}\PY{k+kp}{paste}\PY{p}{(}x\PY{p}{,}\PY{l+s}{\PYZdq{}}\PY{l+s}{manyfriends\PYZdq{}}\PY{p}{,}sep\PY{o}{=}\PY{l+s}{\PYZdq{}}\PY{l+s}{\PYZti{}\PYZdq{}}\PY{p}{)}\PY{p}{)}\PY{p}{,}data\PY{o}{=}survey\PY{p}{)}\PY{p}{)}
          tests
\end{Verbatim}


    
    \begin{verbatim}
[[1]]

	Welch Two Sample t-test

data:  I.use.Facebook.to.find.people.to.date. by manyfriends
t = -1.6922, df = 49.885, p-value = 0.09684
alternative hypothesis: true difference in means is not equal to 0
95 percent confidence interval:
 -0.49228592  0.04209523
sample estimates:
mean in group FALSE  mean in group TRUE 
           1.447368            1.672464 


[[2]]

	Welch Two Sample t-test

data:  I.use.Facebook.to.meet.new.people. by manyfriends
t = -2.9947, df = 48.699, p-value = 0.004309
alternative hypothesis: true difference in means is not equal to 0
95 percent confidence interval:
 -0.8731783 -0.1718255
sample estimates:
mean in group FALSE  mean in group TRUE 
           1.578947            2.101449 


[[3]]

	Welch Two Sample t-test

data:  I.use.Facebook.to.find.people.to.add.to.my..friends..list. by manyfriends
t = 0.12584, df = 45.891, p-value = 0.9004
alternative hypothesis: true difference in means is not equal to 0
95 percent confidence interval:
 -0.3734089  0.4232088
sample estimates:
mean in group FALSE  mean in group TRUE 
           2.076923            2.052023 


[[4]]

	Welch Two Sample t-test

data:  I.use.Facebook.to.learn.more.about.other.people.in.my.classes. by manyfriends
t = -3.0188, df = 42.866, p-value = 0.004262
alternative hypothesis: true difference in means is not equal to 0
95 percent confidence interval:
 -1.2097863 -0.2407095
sample estimates:
mean in group FALSE  mean in group TRUE 
           2.263158            2.988406 


[[5]]

	Welch Two Sample t-test

data:  I.use.Facebook.to.learn.more.about.other.people.living.near.me. by manyfriends
t = -3.3625, df = 43.384, p-value = 0.001622
alternative hypothesis: true difference in means is not equal to 0
95 percent confidence interval:
 -1.1641441 -0.2913984
sample estimates:
mean in group FALSE  mean in group TRUE 
           2.243243            2.971014 


[[6]]

	Welch Two Sample t-test

data:  I.use.Facebook.to.keep.in.touch.with.my.old.friends. by manyfriends
t = -2.9236, df = 40.273, p-value = 0.005655
alternative hypothesis: true difference in means is not equal to 0
95 percent confidence interval:
 -0.7299161 -0.1333030
sample estimates:
mean in group FALSE  mean in group TRUE 
           4.263158            4.694767 


    \end{verbatim}

    
    \begin{Verbatim}[commandchars=\\\{\}]
{\color{incolor}In [{\color{incolor}246}]:} \PY{k+kn}{library}\PY{p}{(}dplyr\PY{p}{)}
          
          survey \PY{o}{\PYZpc{}\PYZgt{}\PYZpc{}} group\PYZus{}by\PY{p}{(}manyfriends\PY{p}{)} \PY{o}{\PYZpc{}\PYZgt{}\PYZpc{}} summarise\PY{p}{(}
              n\PY{o}{=}n\PY{p}{(}\PY{p}{)}\PY{p}{,}
              date.mean\PY{o}{=}\PY{k+kp}{mean}\PY{p}{(}I.use.Facebook.to.find.people.to.date.\PY{p}{,} na.rm\PY{o}{=}\PY{n+nb+bp}{T}\PY{p}{)}\PY{p}{,}
              date.sd\PY{o}{=}sd\PY{p}{(}I.use.Facebook.to.find.people.to.date.\PY{p}{,} na.rm\PY{o}{=}\PY{n+nb+bp}{T}\PY{p}{)}\PY{p}{,}
              meet.mean\PY{o}{=}\PY{k+kp}{mean}\PY{p}{(}I.use.Facebook.to.meet.new.people.\PY{p}{,} na.rm\PY{o}{=}\PY{n+nb+bp}{T}\PY{p}{)}\PY{p}{,}
              meet.sd\PY{o}{=}sd\PY{p}{(}I.use.Facebook.to.meet.new.people.\PY{p}{,} na.rm\PY{o}{=}\PY{n+nb+bp}{T}\PY{p}{)}\PY{p}{,}
              friend\PYZus{}list.mean\PY{o}{=}\PY{k+kp}{mean}\PY{p}{(}I.use.Facebook.to.find.people.to.add.to.my..friends..list.\PY{p}{,} na.rm\PY{o}{=}\PY{n+nb+bp}{T}\PY{p}{)}\PY{p}{,}
              friend\PYZus{}list.sd\PY{o}{=}sd\PY{p}{(}I.use.Facebook.to.find.people.to.add.to.my..friends..list.\PY{p}{,} na.rm\PY{o}{=}\PY{n+nb+bp}{T}\PY{p}{)}
          \PY{p}{)}
\end{Verbatim}


    \begin{tabular}{r|llllllll}
 manyfriends & n & date.mean & date.sd & meet.mean & meet.sd & friend\_list.mean & friend\_list.sd\\
\hline
	 FALSE     &  40       & 1.447368  & 0.7604184 & 1.578947  & 1.003550  & 2.076923  & 1.178416 \\
	  TRUE     & 349       & 1.672464  & 0.9244552 & 2.101449  & 1.165703  & 2.052023  & 1.107371 \\
	    NA     &  48       &      NaN  &        NA &      NaN  &       NA  &      NaN  &       NA \\
\end{tabular}


    
    \begin{Verbatim}[commandchars=\\\{\}]
{\color{incolor}In [{\color{incolor}245}]:} survey \PY{o}{\PYZpc{}\PYZgt{}\PYZpc{}} group\PYZus{}by\PY{p}{(}manyfriends\PY{p}{)} \PY{o}{\PYZpc{}\PYZgt{}\PYZpc{}} summarise\PY{p}{(}
              n\PY{o}{=}n\PY{p}{(}\PY{p}{)}\PY{p}{,}
              classmates.mean\PY{o}{=}\PY{k+kp}{mean}\PY{p}{(}I.use.Facebook.to.learn.more.about.other.people.in.my.classes.\PY{p}{,} na.rm\PY{o}{=}\PY{n+nb+bp}{T}\PY{p}{)}\PY{p}{,}
              classmates.sd\PY{o}{=}sd\PY{p}{(}I.use.Facebook.to.learn.more.about.other.people.in.my.classes.\PY{p}{,} na.rm\PY{o}{=}\PY{n+nb+bp}{T}\PY{p}{)}\PY{p}{,}
              neighbors.mean\PY{o}{=}\PY{k+kp}{mean}\PY{p}{(}I.use.Facebook.to.learn.more.about.other.people.living.near.me.\PY{p}{,} na.rm\PY{o}{=}\PY{n+nb+bp}{T}\PY{p}{)}\PY{p}{,}
              neighbors.sd\PY{o}{=}sd\PY{p}{(}I.use.Facebook.to.learn.more.about.other.people.living.near.me.\PY{p}{,} na.rm\PY{o}{=}\PY{n+nb+bp}{T}\PY{p}{)}\PY{p}{,}
              old\PYZus{}friends.mean\PY{o}{=}\PY{k+kp}{mean}\PY{p}{(}I.use.Facebook.to.keep.in.touch.with.my.old.friends.\PY{p}{,} na.rm\PY{o}{=}\PY{n+nb+bp}{T}\PY{p}{)}\PY{p}{,}
              old\PYZus{}friends.sd\PY{o}{=}sd\PY{p}{(}I.use.Facebook.to.keep.in.touch.with.my.old.friends.\PY{p}{,} na.rm\PY{o}{=}\PY{n+nb+bp}{T}\PY{p}{)}
          \PY{p}{)}
          
          
          tbl \PY{o}{\PYZlt{}\PYZhy{}} \PY{k+kt}{data.frame}\PY{p}{(}
              \PY{c+c1}{\PYZsh{}\PYZsq{}dependent by independent\PYZsq{}=apply(matrix(1:length(tests)), 1, function(i) unname(tests[[i]]\PYZdl{}data.name)),}
              \PY{c+c1}{\PYZsh{}many.friends.n=sum(survey\PYZdl{}manyfriends==TRUE, na.rm=T),}
              \PY{c+c1}{\PYZsh{}few.friends.n=sum(survey\PYZdl{}manyfriends==FALSE, na.rm=T),}
              use.fb.to\PY{o}{=}\PY{k+kp}{names}\PY{p}{(}survey\PY{p}{)}\PY{p}{[}\PY{l+m}{87}\PY{o}{:}\PY{l+m}{92}\PY{p}{]}\PY{p}{,}
              use.to\PY{o}{=}\PY{k+kt}{c}\PY{p}{(}\PY{l+s}{\PYZdq{}}\PY{l+s}{date\PYZdq{}}\PY{p}{,} \PY{l+s}{\PYZdq{}}\PY{l+s}{meet\PYZdq{}}\PY{p}{,} \PY{l+s}{\PYZdq{}}\PY{l+s}{friend\PYZus{}list\PYZdq{}}\PY{p}{,} \PY{l+s}{\PYZdq{}}\PY{l+s}{snoop\PYZus{}classmates\PYZdq{}}\PY{p}{,} \PY{l+s}{\PYZdq{}}\PY{l+s}{snoop\PYZus{}neighbors\PYZdq{}}\PY{p}{,} \PY{l+s}{\PYZdq{}}\PY{l+s}{old\PYZus{}friends\PYZdq{}}\PY{p}{)}\PY{p}{,}
              t\PY{o}{=}\PY{k+kp}{apply}\PY{p}{(}\PY{k+kt}{matrix}\PY{p}{(}\PY{l+m}{1}\PY{o}{:}\PY{k+kp}{length}\PY{p}{(}tests\PY{p}{)}\PY{p}{)}\PY{p}{,} \PY{l+m}{1}\PY{p}{,} \PY{k+kr}{function}\PY{p}{(}i\PY{p}{)} \PY{k+kp}{unname}\PY{p}{(}tests\PY{p}{[[}i\PY{p}{]]}\PY{o}{\PYZdl{}}statistic\PY{p}{[[}\PY{l+s}{\PYZsq{}}\PY{l+s}{t\PYZsq{}}\PY{p}{]]}\PY{p}{)}\PY{p}{)}\PY{p}{,}
              df\PY{o}{=}\PY{k+kp}{apply}\PY{p}{(}\PY{k+kt}{matrix}\PY{p}{(}\PY{l+m}{1}\PY{o}{:}\PY{k+kp}{length}\PY{p}{(}tests\PY{p}{)}\PY{p}{)}\PY{p}{,} \PY{l+m}{1}\PY{p}{,} \PY{k+kr}{function}\PY{p}{(}i\PY{p}{)} \PY{k+kp}{unname}\PY{p}{(}tests\PY{p}{[[}i\PY{p}{]]}\PY{o}{\PYZdl{}}parameter\PY{p}{[[}\PY{l+s}{\PYZsq{}}\PY{l+s}{df\PYZsq{}}\PY{p}{]]}\PY{p}{)}\PY{p}{)}\PY{p}{,}
              p.value\PY{o}{=}\PY{k+kp}{apply}\PY{p}{(}\PY{k+kt}{matrix}\PY{p}{(}\PY{l+m}{1}\PY{o}{:}\PY{k+kp}{length}\PY{p}{(}tests\PY{p}{)}\PY{p}{)}\PY{p}{,} \PY{l+m}{1}\PY{p}{,} \PY{k+kr}{function}\PY{p}{(}i\PY{p}{)} \PY{k+kp}{unname}\PY{p}{(}tests\PY{p}{[[}i\PY{p}{]]}\PY{o}{\PYZdl{}}p.value\PY{p}{)}\PY{p}{)}
              \PY{c+c1}{\PYZsh{}many.friends.est=apply(matrix(1:length(tests)), 1, function(i) unname(tests[[i]]\PYZdl{}estimate[1])), \PYZsh{} redundant with means above}
              \PY{c+c1}{\PYZsh{}few.friends.est=apply(matrix(1:length(tests)), 1, function(i) unname(tests[[i]]\PYZdl{}estimate[2]))}
          \PY{p}{)}
          \PY{c+c1}{\PYZsh{}tbl\PYZdl{}use.fb.to \PYZlt{}\PYZhy{} c(\PYZdq{}date\PYZdq{}, \PYZdq{}meet\PYZdq{}, \PYZdq{}friend\PYZus{}list\PYZdq{}, \PYZdq{}snoop\PYZus{}classmates\PYZdq{}, \PYZdq{}snoop\PYZus{}neighbors\PYZdq{}, \PYZdq{}old\PYZus{}friends\PYZdq{})}
          tbl
          \PY{c+c1}{\PYZsh{}names(fb.use.vars) \PYZlt{}\PYZhy{} c(\PYZsq{}date\PYZsq{},\PYZsq{}meet\PYZsq{},\PYZsq{}friend\PYZus{}list\PYZsq{},\PYZsq{}snoop\PYZus{}classmates\PYZsq{},\PYZsq{}snoop\PYZus{}neighbors\PYZsq{},\PYZsq{}old\PYZus{}friends\PYZsq{})}
\end{Verbatim}


    \begin{tabular}{r|llllllll}
 manyfriends & n & date.mean & date.sd & meet.mean & meet.sd & friend\_list.mean & friend\_list.sd\\
\hline
	 FALSE     &  40       & 1.447368  & 0.7604184 & 1.578947  & 1.003550  & 2.076923  & 1.178416 \\
	  TRUE     & 349       & 1.672464  & 0.9244552 & 2.101449  & 1.165703  & 2.052023  & 1.107371 \\
	    NA     &  48       &      NaN  &        NA &      NaN  &       NA  &      NaN  &       NA \\
\end{tabular}


    
    \begin{tabular}{r|llllllll}
 manyfriends & n & classmates.mean & classmates.sd & neighbors.mean & neighbors.sd & old\_friends.mean & old\_friends.sd\\
\hline
	 FALSE     &  40       & 2.263158  & 1.427230  & 2.243243  & 1.256216  & 4.263158  & 0.8909215\\
	  TRUE     & 349       & 2.988406  & 1.190995  & 2.971014  & 1.202843  & 4.694767  & 0.5583988\\
	    NA     &  48       &      NaN  &       NA  &      NaN  &       NA  &      NaN  &        NA\\
\end{tabular}


    
    \begin{tabular}{r|lllll}
 use.fb.to & use.to & t & df & p.value\\
\hline
	 I.use.Facebook.to.find.people.to.date.                          & date                                                            & -1.6922122                                                      & 49.88490                                                        & 0.096843553                                                    \\
	 I.use.Facebook.to.meet.new.people.                              & meet                                                            & -2.9947010                                                      & 48.69902                                                        & 0.004309117                                                    \\
	 I.use.Facebook.to.find.people.to.add.to.my..friends..list.      & friend\_list                                                   &  0.1258426                                                      & 45.89110                                                        & 0.900406235                                                    \\
	 I.use.Facebook.to.learn.more.about.other.people.in.my.classes.  & snoop\_classmates                                              & -3.0188186                                                      & 42.86636                                                        & 0.004261605                                                    \\
	 I.use.Facebook.to.learn.more.about.other.people.living.near.me. & snoop\_neighbors                                               & -3.3625265                                                      & 43.38369                                                        & 0.001621769                                                    \\
	 I.use.Facebook.to.keep.in.touch.with.my.old.friends.            & old\_friends                                                   & -2.9236079                                                      & 40.27269                                                        & 0.005654626                                                    \\
\end{tabular}


    
    \begin{enumerate}
\def\labelenumi{\alph{enumi})}
\setcounter{enumi}{1}
\tightlist
\item
  Explain any significant differences in how the two groups of users,
  those with more and fewer Facebook friends, are using Facebook by
  reporting the means and standard deviations for the two groups for the
  relevant type of use (e.g. ``Those with more friends are more likely
  to use Facebook to find someone to date (Mean = X, SD = Y) than those
  who have fewer friends (Mean = Z, SD = W)'').
\end{enumerate}

There are significant differences in some uses of Facebook for the two
groups of users but not others.

Finding people to date: low scores for both groups (1.4, 1.7), little
difference between groups (p-value .1) Meet new people: low scores for
both groups (1.6, 2.1), significant difference between groups (p-value:
.004) Add to friend list: low scores for both groups (2.08, 2.05),
little difference between groups (p-value: .9) Learn about classmates:
medium scores for both groups (2.3, 3.0), significant difference between
groups (p-value: .004) Learn about neighbors: low scores for both groups
(2.2, 3.0), significant difference between groups (p-value: .002) Keep
in touch: high scores for both groups (4.3, 4.7), significant difference
between groups (p-value: .006)

Did not calculate Cohen's d for effect sizes, but all of the differences
are smaller than one Likert Scale point, so they do not seem
particularly meaningful.

    \begin{enumerate}
\def\labelenumi{\alph{enumi})}
\setcounter{enumi}{2}
\tightlist
\item
  What would you conclude?
\end{enumerate}

Facebook users with large numbers of friends appear to be more inclined
to use Facebook for keeping in touch with old friends and learning about
people in their classes. To this analyst, these results don't suggest a
fundamental difference in the ways the groups we studied use Facebook. A
plausible explanation for the differences we did see would be that when
people have larger Facebook friend networks, the platform is more useful
for keeping in touch with friends and learning about people in one's
larger community.

    \begin{enumerate}
\def\labelenumi{\alph{enumi})}
\setcounter{enumi}{3}
\tightlist
\item
  What are the limitations, if any?
\end{enumerate}

Rather than dividing into two groups with a cutoff at 150, it might have
been more informative to stratify into more levels and inspect
differences in Facebook use for those with very few or very many
friends.


    % Add a bibliography block to the postdoc
    
    
    
    \end{document}
